   
\documentclass[a4paper,10pt,twocolumn]{article}
\renewcommand{\baselinestretch}{1.05}
\usepackage{amsmath,amsthm,verbatim,amssymb,amsfonts,amscd, graphicx}
\usepackage{graphics}
\usepackage{multicol}
\topmargin0.0cm
\headheight0.0cm
\headsep0.0cm
\oddsidemargin0.0cm
\textheight23.0cm
\textwidth16.5cm
\footskip1.0cm
\theoremstyle{plain}
\newtheorem{theorem}{Theorem}
\newtheorem{corollary}{Corollary}
\newtheorem{lemma}{Lemma}
\newtheorem{proposition}{Proposition}
\newtheorem*{surfacecor}{Corollary 1}
\newtheorem{conjecture}{Conjecture} 
\newtheorem{question}{Question} 
\theoremstyle{definition}
\newtheorem{definition}{Definition}

 \begin{document}
 


\title{Sparse Neural Nets: Improving Feedforward Neural Networks Efficiency with Sparse Matrix Constructions}
\author{Dr. Choromanski, \textit{Google Research}\\ Bourely, Alfred\textit{Columbia University}\\ Boueri, Patrick \textit{Uptake Technologies}}
\maketitle

\section{Abstract}
We show empirical evidence that sparse connectivity in between layers in a Feed Forward Neural Network does not impact accuracy significantly, as compared to a fully connected layer. Using the canonical MNIST data set, we compute accuracy measures for many Feed Forward Neural Nets with different connection schemes and topologies, showing there is no significant drop off as low as 10% of connections in the fully connected setting. With sparse connection paradigms, neural networks can be stored with less memory and predictions can be made with faster or with less computation resources, suggesting suitable applications in IoT. 


\subsection{Math Mode}\label{section:mathmode}
% If you're wondering about the ``\label" above, it will be explained below.  Note that this line of text which follows the percent sign doesn't show up in the pdf.  This is a good way to leave notes for yourself on a work in progress.
Some formatting can be done in text mode (for example, you can make the font \textit{italic} or \textbf{boldface}), but for most mathematical symbols, you'll have to use math mode.  Math mode is most often introduced and ended with a \$.  For example, in math mode I can write the equation $x+ y=7$ and the program takes care of spacing.  It's also easy to write Greek letters ($\alpha$, $\Sigma$), exponents ($2^{x+y}$), and subscripts ($x_1$).  Look in any LaTeX guide to find a list of symbols and formatting commands.  

\section{References}
One of the nice things about using LaTeX is that it makes internal references easy.  For example, if I want to remind you where I discussed math mode, I can mention that it was in Section~\ref{section:mathmode}.  If you're looking at the pdf file, you see the correct reference, but in the TeX file I typed a label that I had attached to that section.  (You may need to typeset your document more than once to make the references show up correctly.)  Labels work for definitions, theorems, questions, sections, diagrams, and equations, among others.

 
\end{document}